\section{Задание}

\textbf{Вариант: 200} \\
Внимание! У разных вариантов разный текст задания!\\
Разделить программу из лабораторной работы №5 на клиентский и серверный модули. Серверный модуль должен осуществлять выполнение команд по управлению коллекцией. Клиентский модуль должен в интерактивном режиме считывать команды, передавать их для выполнения на сервер и выводить результаты выполнения.\\

\textbf{Необходимо выполнить следующие требования:}\\
- Операции обработки объектов коллекции должны быть реализованы с помощью Stream API с использованием лямбда-выражений.\\
	- Объекты между клиентом и сервером должны передаваться в сериализованном виде.\\
	- Объекты в коллекции, передаваемой клиенту, должны быть отсортированы по имени\\
	- Клиент должен корректно обрабатывать временную недоступность сервера.\\
	- Обмен данными между клиентом и сервером должен осуществляться по протоколу UDP\\
	- Для обмена данными на сервере необходимо использовать \textbf{сетевой канал}\\
	- Для обмена данными на клиенте необходимо использовать \textbf{датаграммы}\\
	- Сетевые каналы должны использоваться в неблокирующем режиме.\\
	
\textbf{Обязанности серверного приложения:}\\
- Работа с файлом, хранящим коллекцию.\\
	- Управление коллекцией объектов.\\
	- Назначение автоматически генерируемых полей объектов в коллекции.\\
	- Ожидание подключений и запросов от клиента.\\
	- Обработка полученных запросов (команд).\\
	- Сохранение коллекции в файл при завершении работы приложения.\\
	- Сохранение коллекции в файл при исполнении специальной команды, доступной только серверу (клиент такую команду отправить не может).\\
	
\textbf{Серверное приложение должно состоять из следующих модулей (реализованных в виде одного или нескольких классов):}\\
	- Модуль приёма подключений.\\
	- Модуль чтения запроса.\\
	- Модуль обработки полученных команд.\\
	- Модуль отправки ответов клиенту.\\
	- Сервер должен работать в \textbf{однопоточном} режиме.\\
	
\textbf{Обязанности клиентского приложения:}\\
	- Чтение команд из консоли.\\
	- Валидация вводимых данных.\\
	- Сериализация введённой команды и её аргументов.\\
	- Отправка полученной команды и её аргументов на сервер.\\
	- Обработка ответа от сервера (вывод результата исполнения команды в консоль).\\
	- Команду save из клиентского приложения необходимо убрать.\\
	- Команда exit завершает работу клиентского приложения.\\
	
\textbf{Важно!} Команды и их аргументы должны представлять из себя объекты классов. Недопустим обмен "простыми" строками. Так, для команды add или её аналога необходимо сформировать объект, содержащий тип команды и объект, который должен храниться в вашей коллекции.\\

\textbf{Дополнительное задание:}\\
- Реализовать логирование различных этапов работы сервера (начало работы, получение нового подключения, получение нового запроса, отправка ответа и т.п.) с помощью \textbf{Log4J2}\\

\textbf{Отчёт по работе должен содержать:} \\
	1. Текст задания.\\
	2. Диаграмма классов разработанной программы (как клиентского, 			3. так и серверного приложения).\\
	4. Исходный код программы.\\
	5. Выводы по работе.\\
	
\textbf{Вопросы к защите лабораторной работы:}\\
	1. Сетевое взаимодействие - клиент-серверная архитектура, основные протоколы, их сходства и отличия.\\
	2. Протокол TCP. Классы Socket и ServerSocket.\\
	3. Протокол UDP. Классы DatagramSocket и DatagramPacket.\\
	4. Отличия блокирующего и неблокирующего ввода-вывода, их преимущества и недостатки. Работа с сетевыми каналами.\\
	5. Классы SocketChannel и DatagramChannel.\\
	6. Передача данных по сети. Сериализация объектов.\\
	7. Интерфейс Serializable. Объектный граф, сериализация и десериализация полей и методов.\\
	8. Java Stream API. Создание конвейеров. Промежуточные и терминальные операции.\\
	9. Шаблоны проектирования: Decorator, Iterator, Factory method, Command, Flyweight, Interpreter, Singleton, Strategy, Adapter, Facade, Proxy.\\

\section{Код и Диаграмм}
\textbf{Исходный код доступен по ссылке или QR-коду:}\\
\\     
\underline{$https://github.com/ndwannafly/Programming-Lab-2nd-Semester/tree/main/LAB_6$}\\


\begin{figure}[H]
\includegraphics[scale=0.8]{img/SourceCode}
\label{pic:SourceCode}
\end{figure}

\underline{$https://github.com/ndwannafly/Programming-Lab-2nd-Semester/blob/main/LAB_6/report$}\\


\begin{figure}[H]
\includegraphics[scale=0.8]{img/QRDiagram}
\label{pic:QRDiagram}
\end{figure}

\section{Вывод}
В ходе этой лабораторной работы мы знаем о Client-Server архитектуры, работать с сокетами, каналами, интерфейсами Serializable и Stream API для обработки коллекций. Кроме того, мы знали, как проектировать неблокирующий ввод-вывод.